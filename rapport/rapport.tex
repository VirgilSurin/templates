\documentclass[12pt,a4paper, titlepage]{article}

\usepackage{times}
\usepackage[frenchb]{babel}
\usepackage{hyperref} 
\usepackage[utf8]{inputenc}
\usepackage[T1]{fontenc}
\usepackage{url}
\usepackage{amsmath}
\usepackage{amsfonts}
%\usepackage{amscd}
\usepackage{amstext}
\usepackage{amssymb}
%\usepackage{bar}
\usepackage{color}
%\usepackage{mathrsfs}
\usepackage{graphicx}
%\usepackage{calligra}
\usepackage{amsthm}
%\usepackage{multirow}
\usepackage{tabularx}
%\usepackage{layout}
\usepackage{url}
%\pagestyle{headings}
\usepackage{fancyhdr}

\newtheorem{defi}{D\'efinition}[section]
\newtheorem{note}{Note}[section]
\newtheorem{proprietet}{Propri\'et\'e}[section]
\newtheorem{exemple}{Exemple}[section]
\newtheorem{corollaire}{Corollaire}[section]
\newtheorem{rem}{Remarque}[section]
\newtheorem{thm}{Th\'eor\`eme}[section]
\newtheorem{illustration}{Illustration}[section]
\newenvironment{demonstration}{\begin{proof}[\textnormal{\textbf{Preuve.}}]}{\end{proof}}
\definecolor{gris}{gray}{0.45}
\setlength{\parindent}{1cm}
\newcommand{\textcalli}[1]{{\small{\textbf{$\negmedspace$\calligra #1}}}}

\fancyhf{} % supprime les en-têtes et pieds prédéfinis
%\fancyhead[RE]{\textsl{\leftmark}} % Right Even
\renewcommand{\headrulewidth}{0pt}% filet en haut de page
\renewcommand{\footrulewidth}{0pt} % pas de filet en bas
\fancypagestyle{plain}{ % pages de tetes de chapitre
\fancyhead{} % supprime l’entete
\fancyhead[R]{\thepage}
\renewcommand{\headrulewidth}{0pt} % et le filet
}

\begin{document}

%newpage
%\thispagestyle{empty}
%\null
%\newpage

\begin{titlepage}
  \begin{center}
    \textnormal{\Large{Universit\'e de Mons}}\\[0.3em]
    \textnormal{\Large{Facult\'e des Sciences}}\\[0.3em]
    \textnormal{\Large{D\'epartement d'Informatique}}\\[0.3em]
    \textnormal{\Large{NOM DU SERVICE (optionnel)}}
  \end{center}
  \vspace*{3cm}
  \begin{center}
    \fbox{
      \begin{minipage}{0.9\textwidth}
        \centering
        \vspace*{0.5cm}\textbf{\LARGE{Intitulé}}\\[0.5em]
        \textbf{\LARGE{suite de l'intitul\'e}}\vspace*{0.5cm}
      \end{minipage}
    }
  \end{center}
  \vspace*{2cm}

  \begin{minipage}[t]{0.45\textwidth}
    \begin{flushleft} \large
      \emph{Directeur:}\\
      Foo \textsc{Bar}\\
      \emph{Rapporteurs:}\\
      John \textsc{Doe}\\
    \end{flushleft}
  \end{minipage}
  \begin{minipage}[t]{0.45\textwidth}
    \begin{flushright} \large
      \emph{Auteurs:} \\
      Alice \textsc{Andbob} \\
      Joe \textsc{Biden} \\
    \end{flushright}
  \end{minipage}\\[2ex]

  \vspace*{2cm}
  \begin{center}
    \includegraphics[height=2cm]{images/UMONS-logo.jpg}
    \hspace{3cm}
    \includegraphics[height=1.7cm]{images/FS-logo.jpg}
    \\[1em]
    Ann\'ee acad\'emique 2013-2024
  \end{center}

\end{titlepage}
\newpage
% \include{pageGarde}
\section*{Remerciements}
%\addcontentsline{toc}{chapter}{Remerciements}

Nous remercions ...\\

\newpage
\thispagestyle{fancy}
\tableofcontents


\newpage
\section*{Introduction}
\addcontentsline{toc}{section}{Introduction}
Mettez l'introduction ici. Expliquez le contexte du travail, et les objectifs du travail.

Avant de commencer la r\'edaction d'un projet ou un mémoire, lisez d'abord attentivement les conseils donn\'es dans \cite{Melot2007UMONS}.

\newpage

\section{Section}%
\label{sec:1}

Ici on a la premi\`ere section : ~\ref{sec:1}.

\subsection{Une sous section}

\subsection{Encore une sous section}

\section{Encore une section}


%ICI COMMENCE LE DERNIER CHAPITRE
\section*{Conclusion}
\addcontentsline{toc}{section}{Conclusion}

Mettez votre conclusion ici.  Dressez le bilan de votre travail effectué, en prenant du recul. Discuter de si vous avez bien réussi les objectifs du travail ou non. Présentez les perspectives futurs.


%Le style bibliographique utilisŽ
\bibliographystyle{latex8}

%Le fichier .bib uitilisŽ
\bibliography{biblio}

\end{document}
